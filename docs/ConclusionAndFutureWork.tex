\chapter{Conclusion and future work}
\label{chap:ConclusionAndFutureWork}
Our application can successfully render every building from Eindhoven, every element with grass and water and the roads. The objects have textures drawn on them to look appealing. Also different kinds of roads have different textures. The Buildings are generated from the BAG dataset and the other elements are generated from OpenStreetMap data set.

The data is rendered in real-time. Our tests show that a HLOD can make a big difference in rendering this kind of data. The GPU load is about 1.7 times lower with some error introduced, while the differences are not visible on the screen. The memory difference isn’t great. We assume that this is because the data set is relative small. Our single node data set is only 170MB. When the size increases, then the dataset does not fit in the GPU memory and then it will probably show that the memory usage is still very low.

The error that we introduce, which causes simplified versions of buildings to be rendered, is visible in some cases. This is only so little that most of the time this is unnoticeable. When there is a lot of occlusion then this error can even be increased, so the GPU has even less work.

The application can be improved by an algorithm that gives better result when simplifying the models. Also more data can be imported from OpenStreetMap. At the moment, all of our buildings have the same texture. The visualisation can be made even more appealing by applying different textures to buildings. This could be done based on the type of a building and the build year. Also more functionality could be introduced. Now, users can only fly or walk through the world to see what it looks like. The possibility to get information on buildings and the surrounding area would make the application more interesting.
