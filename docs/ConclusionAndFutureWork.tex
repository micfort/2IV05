\chapter{Conclusion and future work}
\label{chap:ConclusionAndFutureWork}
Our application can successfully render every building from the Netherlands, every element with grass and water and the roads. The objects have textures drawn on them to look appealing. Also different kinds of roads have different textures. The Buildings are generated from the BAG dataset and the other elements are generated from OpenStreetMap data set.

The data is rendered in real-time. Our Netherlands data set is around 24.1 GB and this can be rendered with a PC that only has 1 GB GPU memory and 8 GB CPU memory. When the error settings are larger, more is loaded in too memory, but the distance can be even more improved. The distance is almost 10 folded while still maintaining a manageable memory usage.

In figure \ref{fig:Simplification} it is shown that error on the screen is introduced by the simplification done to buildings. This simplification can be seen on the screen, but is not extremely annoying. It works intuitive to get more detail when the viewpoint is closer.

The application can be improved by an algorithm that gives better result when simplifying the models, so that the difference in level of details is less. Also more data can be imported from OpenStreetMap. At the moment, all of our buildings have the same texture. The visualisation can be made even more appealing by applying different textures to buildings. This could be done based on the type of a building and the build year. To make buildings more appealing, windows and other features can also be added. The possibility to get information on buildings and the surrounding area would make the application more interesting.
