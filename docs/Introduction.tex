\chapter{Introduction}
This report presents the results of the project “Visualizing the Netherlands” as part of the course “Additional component computer graphics”. The goal of the project was to design and implement a system which is able to visualize the Netherland based on the BAG-extract [1] and other open data sources. The BAG-extract is a data set that contains information on all the buildings and addresses in the Netherlands.
%change reference

Firstly, the report presents the open data sets used, then the requirements and the desired functionality of the system is described. In section 4 the problems and challenges of the project are described shortly. Related work, which helped us solving the project’s challenges, is presented in section 5. In the literature study, methods to construct and visualize large worlds out of large data sets have been researched. Then in section 6, a short analysis of the used data sets is made and the system design and algorithms used are presented. The system uses specific data structures and methods to access the out-of-memory data set. Algorithms to construct and to use these data structures are also presented. Further, in section 7 the results of the implementation of the designed system and algorithms are discussed. Here the implemented system will be tested in various scenarios to see how the system performs. Finally a conclusion on the results is made and possible improvements or extensions on the systems are proposed.
