\chapter{Requirements}
The application visualizes the Netherlands on the screen in 3D. All the buildings that are in the data set are being rendered. There are datasets for the whole of the Netherlands. It is possible to fly over an area of the Netherlands and walk through cities. Besides buildings, also roads and ground type (grass, farm land, water and industrial terrain) are rendered.
Because of the immensely huge data set of buildings in the Netherlands, this data set is filtered, preprocessed and saved in an efficient format, so that it can be used by the algorithms to construct the world. OpenStreetMap has data about the streets, ground type and trees. That data set is also used to render the world.

\section{Specification of functionality}
The following functionality is implemented in the system.
\begin{itemize}
  \item Read data from the BAG data set.\\
    The BAG data set is quite large. The system needs to be able to read and filter the data from the BAG such that the required data can be retrieved.
  \item Construct buildings from geographic data. \\
    Models of buildings need to be constructed from the data read from the BAG. The model of a building has to be modeled to look like the corresponding real building.
  \item Construct roads and surface area.\\
    To make cities and other areas look like the real world, also roads and surface areas like grass, rivers and lakes can be constructed. These elements can contribute highly to make areas recognizable.
  \item Display the constructed world in 3D in real time.\\
    Since rendering a whole area, like Noord Brabant, can be quite large, an efficient rendering algorithm has to be designed or used so that the application can be rendered in real time.
\end{itemize}

\section{Specification of interaction}
There are 2 different modes to change the view. The first view is that you fly through and over a city or landscape. In the second view, the camera of the system will be set to a height around 1.75m. That way you see the world as any human would see it as he or she would walk through the city. From the start, the user is able to select a city he or she would like to visit. The user is then placed somewhere above or within the city and is able to go anywhere the user wants. Also the speed in which the user is walking or flying can be configure so that the user is able to walk quickly though the city or is able to fly towards another city.

\section{Specification of presentation}
The world will be rendered dependent on the position of the user. If the user is inside a city, then all buildings in the proximity of the user will be rendered with high detail. While flying the user is able to go to a higher altitude. Since buildings and other parts will become very small, a lower level of detail can be applied to the elements which have to be rendered. When the user gets even higher then multiple buildings will be combined into one model. At the top level, only the landscape will be rendered. 